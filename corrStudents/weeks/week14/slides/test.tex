\documentclass[professionalfonts]{beamer}
\usepackage{polyglossia}
\setdefaultlanguage{russian}
\setotherlanguage{english}
\defaultfontfeatures{Ligatures={TeX},Renderer=Basic}  %% свойства шрифтов по умолчанию. Для XeTeX опцию Renderer=Basic можно не указывать, она необходима для LuaTeX
\setmainfont[Ligatures={TeX,Historic}]{CMU Serif} 
\setsansfont{CMU Sans Serif}
\setmonofont{CMU Typewriter Text}
\usefonttheme{professionalfonts}
\usepackage{unicode-math}
\setmathfont{texgyrepagella-math.otf}
\usetheme{Pittsburgh}
\usecolortheme{owl}
\usepackage{tikz}
\begin{document}
\begin{frame}
	\frametitle{Властелин разума}
	\begin{center}
\begin{tikzpicture}
\filldraw[red]  (0,0)  circle (20pt);
\filldraw[blue] (2, 0)  circle (20pt);
\filldraw[green] (4, 0)  circle (20pt);
\filldraw[yellow] (6, 0)  circle (20pt) node[right=30pt, white] {\Huge 2};
	
\filldraw[blue]  (0,-2)  circle (20pt);
\filldraw[red] (2, -2)  circle (20pt);
\filldraw[green] (4, -2)  circle (20pt);
\filldraw[yellow] (6, -2)  circle (20pt) node[right=30pt, white] {\Huge 0};

\filldraw[red]  (0,-4)  circle (20pt);
\filldraw[blue] (2, -4)  circle (20pt);
\filldraw[yellow] (4, -4)  circle (20pt);
\filldraw[green] (6, -4)  circle (20pt) node[right=30pt, white] {\Huge 4};

\end{tikpicture}
\end{center}
\end{frame}
\end{document}
